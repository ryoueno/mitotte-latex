\chapter{実験}
本章では,提案するシステムの有用性を検証する実験について述べる.

\section{目的}
この実験の目的は,ユーザと管理者両方の視点から有用性を測る.
まず,ユーザがシステムを利用を利用することによって,作業の生産性が向上するかどうかを検証する.
また,管理者は,自動的に報告されるユーザの作業状況を確認し,現状を把握できるかどうかを確認する.

\section{対象}
実験対象者は20代大学生及び,研究室の教授の10名とした.
実験は11月3日から実施した.

\subsection{作業の定義}
今回の実験では,研究室に所属する大学生の論文の執筆を監視する.
論文の執筆は,各個人のPCを利用し,文章作成用のソフトウェアを利用する.
したがって,文章作成用のソフトウェアを起動し,文字を入力することが今回の作業の定義とする.
その際,PCからインターネットを利用して情報を収集することも作業の範囲内とする.
ただし,アニメ動画やゲームなど,明らかに研究と関連しないものを閲覧する行為は作業の範囲外とする.

作業を検出する準備として,予め作業状態を表す情報を収集した.
まず,各個人が作業をしている際のPCの画面キャプチャ画像を50枚集め,解析を行った.
表\ref{tab:positive_words}は,解析によって得られた情報のうち,複数回出現したものである.

\begin{table}[htb]
  \begin{center}
    \begin{tabular}{lll}
      text & web page & website \\
      software & technology & product  \\
      black & computer program & Debian GNU/Linux \\
      font & screenshot & line \\
      black and white & &
    \end{tabular}
    \caption{複数回検出された情報}
    \label{tab:positive_words}
  \end{center}
\end{table}

画像キャプチャ解析において,これらに含まれる範囲の情報を検出した場合は,作業に必要な画面を表示していると判定する.
これら以外の情報を検出した場合は,作業範囲外であると判定する.

\section{方法}
実験では,対象者にシステムを利用しながら論文の執筆を行わせた.

\subsection{実験手順}
監視対象であるユーザは,論文を執筆する際にアプリケーションを起動する.
起動中のアプリケーションから収集される情報をもとに,作業記録の生成を行う.
生成された作業記録は,次の日の朝に,チャットサービスを通じて管理者に送信される.
それらの一連の流れを確認することで,システム自体の信頼性や情報の適正性を測る.

本システムに対するユーザ及び監督者の評価は,アンケート調査によって行う.
構築した監視システムの有用性や,実用的であるかどうかを調査する.

\section{結果}
本システムを2ヶ月間運用し,得られた結果を示す.
まず,監視システムによりユーザの論文執筆の作業を監視した結果を述べる.
その後,ユーザと監督者それぞれの視点から,本システムの評価を受けた結果を述べる.

\subsection{監視システムの運用}
本システムの運用は,作業予定の入力・監視・報告までユーザ及び監督者のみで行われる.
作業予定の入力を行うのはユーザのみであり,予定の追加や変更,削除など全て正常に動作した.
また,アプリケーションによる監視機能も自動的に行われ,PCからサーバへのデータの送受信も正確に行われた.

作業に不要なサービス・アプリケーションは,運用中に970回検出された.
検出されたのはAmazonやYahoo JapanなどのWebサービスや,SkypeやSlackなどPC上のアプリケーションであった.
他にも,ユーザのデスクトップに設定された画像なども作業範囲外の状態として検出された.

作業状況の報告は,処理時間により必ず定時に送信することはできなかったが,おおむね午前中の時間帯に自動的に行われた.
作業をしていた時間帯,作業日時の変更,不正なサービスの検出など,システムで設定された項目は全て記録され,正常に送信された.
作業時間の通知機能も,事前に設定した作業予定時刻に従って正常に動作することが確認できた.

\subsection{ユーザによる本システムの評価}
作業を行う立場のユーザの視点から本システムを評価するためのアンケートは,監視対象であった9名ユーザを対象に実施した.
各質問項目と,回答状況から,本システムに対する評価を述べる.

\subsubsection{本システムにより作業をやらないといけない気になったか}
9名全員が「なった」あるいは「少しなった」と回答した.
本システムにより,作業を行う動機付けが可能であり,行動を促すことを期待できることが分かった.

\subsubsection{本システムを使うことによって生産性は向上すると思うか}
9名全員が「とても思う」あるいは「思う」と回答した.
本システムを利用することによって,作業者の生産性の向上が期待できるという評価がされた.

\subsubsection{自身の作業が報告されることに抵抗があるか}
7名が「抵抗がある」と回答し,2名が「あまり抵抗がない」と回答した.
このことから,作業が監視されることに抵抗を感じるユーザは少なくないと考えられる.
監視システムを構築する上で,監視対象であるユーザの抵抗感をなくしていくことが今後の課題として挙げられる.

\subsubsection{作業を行う際,本システムに監視されていることを意識したか}
7名が「とても意識した」あるいは「意識した」と回答し,2名が「あまり意識していない」と回答した.
意識するユーザが多かったことから,システムにより,作業を開始する際だけでなく,作業中も影響をあたえられることが分かった.

\subsubsection{予定を管理するにあたって,本システムは使いやすいと思うか}
1名が「使いやすい」と回答し,8名が「やや使いにくい」と回答した.
今回のシステムでは,作業内容と日時を入力する簡素な構造であったため,正しく予定を管理するためには,個人が管理するカレンダーと連動させるなど,工夫しなければならない.
また,スマホで管理できるようにしてほしいなど,より手軽さを求める意見も見られた.

\subsubsection{作業開始時間の通知は役に立ったか}
7名が「役に立った」と回答し,2名が「あまり役に立たなかった」と回答した.
作業開始時間の通知機能は,利用者にとってある程度有効であることが分かった.

\subsubsection{通知の頻度は適切であったか}
3名が「適切である」と回答し,6名が「多い」あるいは「とても多い」と回答した.
今回は作業時間から30分毎に繰り返し通知を行うように設定を行ったが,頻度が多い感じたユーザが多く見られた.
ただし,作業の開始を促すための通知であるため,多いと感じる程度が適切であると考える.

\subsubsection{システムによって進捗管理を自動化することは現実的だと思うか}
8名が「現実的である」と回答し,1名が「現実的ではない」と回答した.
「現実的である」と回答した人の意見として,「利用してみて可能だと考えた」というものや,「進捗管理は人間よりもシステムがやったほうが効率的である」というものが挙げられた.
それに対し「非現実的である」と回答した人の意見としては,「プライバシーの面で実現が難しいのではないか」というものが挙げられた.

\subsubsection{その他,システムに関する意見}
本システムを利用することによって,「やらないといけないという気持ちになった」という意見が複数あった.
他にも実績が可視化されることがモチベーションにつながるなど,報告される情報そのものに価値を感じているユーザも見られた.
また,「オフライン環境で動作することができない」「予定の入力が大変である」といった意見もあり,システム自体をより手軽なものにしてほしいという要望が寄せられた.

\subsection{監督者による本システムの評価}
監督者の立場から本システムを評価するためのアンケートは,実際に本システムを利用してユーザを監督した1名の教授を対象に行った.
以下はアンケートの各質問項目と,その回答である.

\subsubsection{本システムにおいて有効だった点}
実際に作業している時間がわかることによって,あとどれぐらいの時間やるべきなのかを具体的に指導することができた.

\subsubsection{本システムにおいて不足していた点}
動かしていない場合などの記録が得られないこと.
作業をする予定でして何もいない場合と,作業をする予定で別のことをしている場合の区別ができないように思える.
作業をしない予定で作業をしていない場合も区別出来た方が,より具体的な管理につながると考えられる.

\subsubsection{システムによって進捗管理を自動化することは現実的だと思うか}
職場での利用を考えると現実的であるが,学生の実態では非現実的かと思われる.
職場では業務時間内に他の個人的なことや連絡にPCを利用することはないが,学生個人のPCでは作業をしながら別の個人的なことを行うのが一般的と想定できるためである.
ただし,学生にも集中して作業する時間を確保するように十分な指導ができれば,現実的であると考えられる.

\subsubsection{その他,システムに関する意見}
本システムを学生に利用してもらってわかったのは,個人で進捗管理ができていないことである.
具体的には,多くの学生は,タスクの詳細化とそれを消化する時間の設定,週や日ごとの消化しきれなかったタスクの見積もりと再配置ができていない.
現状のシステムでも可能だが,そもそもTODO管理が出来ないために,有効活用できていない.
管理ユーザーとしては,管理者ユーザ側が各自のタスクとタスク詳細化と時間の見積もりに関与し,それを各自に割りふりさせ,状況を確認するようなシステムにした方が,より確実に進捗管理が可能となる.

一方,管理ユーザの手を煩わせること無く,管理ユーザに通知されていることを意識することによるなまけ抑止を期待するのであれば,システムは現状のままでも十分である.ただし,前述の通り現状のタスク消化・作業実施状況から,利用ユーザはTODO管理のコツが理解できておらず,最低限そのレクチャーを行う必要がある.

\section{考察}
本システムについて,有用性を検証するために行った実験の考察を述べる.
進捗監視を行うにあたって,実際にユーザが作成した予定を元に作業時間を記録し,自動的に報告を行うことができた.
これにより,ユーザが作業を行う動機付けがなされ,行動を促すことができた.
また,報告される情報から,監督者の立場からもより具体的に作業者に指導を行うことが可能となった.
以上のことより,本システムの有用性が高いことが検証された.
