\chapter{実験}
本章では,提案するシステムの有用性を検証する実験について述べる.

\section{目的}
この実験の目的は,ユーザと管理者両方の視点から有用性を測る.
まず,ユーザがシステムを利用を利用することによって,作業の生産性が向上するかどうかを検証する.
また,管理者は,自動的に報告されるユーザの作業状況を確認し,現状を把握できるかどうかを確認する.

\section{対象}
実験対象者は20代大学生及び,研究室の教授の10名とした.
実験は11月3日から実施した.

\subsection{作業の定義}
今回の実験では,研究室に所属する大学生の論文の執筆を監視する.
論文の執筆は,各個人のPCを利用し,文章作成用のソフトウェアを利用する.
したがって,文章作成用のソフトウェアを起動し,文字を入力することが今回の作業の定義とする.
その際,PCからインターネットを利用して情報を収集することも作業の範囲内とする.
ただし,アニメ動画やゲームなど,明らかに研究と関連しないものを閲覧する行為は作業の範囲外とする.

\section{方法}
実験では,対象者にシステムを利用しながら論文の執筆を行わせた.

\subsection{実験手順}
監視対象であるユーザは,論文を執筆する際に,アプリケーションを起動する.
起動中のアプリケーションから収集される情報をもとに,作業記録の生成を行う.
生成された作業記録は,次の日の朝に,チャットサービスを通じて管理者に送信される.
それらの一連の流れを確認することで,システム自体の信頼性や情報の適正性を測る.

\section{結果}

\section{考察}

%以降削除すること
\clearpage
\noindent
一二三四五六七八九零一二三四五六七八九零一二三四五六七八九零一二三四五\\
二\\
三\\
四\\
五\\
六\\
七\\
八\\
九\\
零\\
一\\
二\\
三\\
四\\
五\\
六\\
七\\
八\\
九\\
零  行数と列数の設定テスト 30行×35文字 = 1050文字/ページ\\
一\\
二\\
三\\
四\\
五\\
六\\
七\\
八\\
九\\
零

