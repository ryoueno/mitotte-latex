\chapter{はじめに}
本論文では,PCワークに限り,進捗管理と行動監視を行い,目標達成を支援するシステムについて記述する.

まず,本研究をおこなう背景となった事柄について述べる.
次に,研究目的の詳細を記述した後,類似研究との相違や関連研究とのつながりについて解説する.
また,次章以降の本論文の構成についてその概略を述べる.

\section{背景}
現在,勉強やオフィスワークでもPCを始めとするデジタル機器を利用することが一般的となっている.
それにより,今まで一般的であったオフィスなど,ひとつの拠点に集まって仕事をするのではなく,ひとりひとりが拠点を離れて仕事をすることも可能となった.在宅勤務は,育児や介護などのために自宅を離れられない個人にとって,家庭生活と仕事を両立するための手段として期待されている.場所の制約をなくすことは,出勤時間そのものも削減することができ,効率的に時間を使うことができる利点がある.

一方,そのスケジュール管理が自分自身に任されていることがデメリットになる場合もある.仕事量を計算し,自己管理できなければ,リモートワークは成立できない.労働時間についても程度勤務者の裁量にゆだねられる.そのため企業にとっては労働時間の管理が難しくなっている.

最近では,企業はグローパルな競争に直面しており,コスト削減を行うことが不可欠になっている.その一環として,リモートワークを活用することも増えてきている.したがって,個人の遠隔地における作業を適切に管理し,生産性を維持していく必要があると考えた.

\section{研究目的}

\section{関連研究}

\section{論文構成}

%以降削除すること
\clearpage
\noindent
一二三四五六七八九零一二三四五六七八九零一二三四五六七八九零一二三四五\\
二\\
三\\
四\\
五\\
六\\
七\\
八\\
九\\
零\\
一\\
二\\
三\\
四\\
五\\
六\\
七\\
八\\
九\\
零  行数と列数の設定テスト 30行×35文字 = 1050文字/ページ\\
一\\
二\\
三\\
四\\
五\\
六\\
七\\
八\\
九\\
零

