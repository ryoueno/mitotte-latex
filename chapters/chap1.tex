\chapter{はじめに}
本論文では,PCワークにおいて,進捗管理と行動監視を行い,目標達成を支援するシステムについて記述する.

まず,本研究をおこなう背景となった事柄について述べる.
次に,研究目的の詳細を記述した後,類似研究との相違や関連研究とのつながりについて解説する.
また,次章以降の本論文の構成についてその概略を述べる.

\section{背景}
現在,勉強やオフィスワークでもPCを始めとするデジタル機器を利用することが一般的となっている.
それにより,今まで一般的であったオフィスなど,ひとつの拠点に集まって仕事をするのではなく,ひとりひとりが拠点を離れて仕事をすることも可能となった.
リモートワークは,育児や介護などのために自宅を離れられない個人にとって,家庭生活と仕事を両立するための手段として期待されている.
場所の制約をなくすことは,出勤時間そのものも削減することができ,効率的に時間を使うことができる利点がある\cite{tele2017}.

一方,リモートワークにおいて,スケジュール管理が自分自身に任されていることがデメリットになる場合もある.
仕事量を計算し,自己管理できなければ,リモートワークは成立できない.
労働時間に関しても,ある程度は勤務者の裁量にゆだねられる.
そのため企業にとっては労働時間の管理が難しくなっている\cite{Adachi2010}.

最近では,企業は国際的な競争に直面しており,コスト削減を行うことが不可欠になっている.
その一環として,リモートワークを活用することが増えてきている\cite{Telework2010}.
以上のことから,個人の遠隔地における作業を適切に管理し,生産性を維持していく必要があると考えた.

\section{研究目的}
本研究の目標は,個人のPCワークの進捗管理と行動監視を行い,生産性を向上させることである.
進捗監視システムにより,個人の作業を管理し,その進捗状況を可視化する.
定期的に進捗状況の確認が行われることは,作業への動機付けとなり,行動を促すことにつながる.
監督者は,作業者の状況を把握し,適切な対応ができるようになる.

本研究の目的は,進捗監視システムを構築し,有用性を検証することである.
個人のPCワークを監視し,進捗報告までを自動的に行う仕組みが,作業者の生産性の向上に貢献できるのかを明らかにする.
そのためにも,必要な情報が適切に扱われる信頼性の高いシステムを構築する必要がある.

本論文では,構築したシステムの具体的な内容と,そのシステムを利用した実験により得られた結果及び考察を述べる.

\section{関連研究}
この節では,本研究と関連している監視システムを利用した研究を2つ紹介する.
次に,関連研究と比較した本研究の特徴について述べる.

\subsection{オンラインテストにおける不正行為の防止策}
PCの監視システムを利用した研究として,オンラインテストにおける不正行為を防止に役立てたものが挙げられる\cite{Hari2006}.
この研究では,監視対象のPCに特定のプログラムをダウンロードし,不正(試験に不必要なもしくは起動が許可されていない)プロセスを監視する.
構築したシステムにより,カンニングなどの不正行為を検出することで,オンライン試験の課題解決を図った.

起動プロセス監視方式には,Webブラウザなどのアプリケーションの追加機能として,アプリケーションをダウンロードさせるエージェント送り込み方式を採用している.
監視用のアプリケーションは,受験者のPC内でプロセスの監視を行う.
試験中に,不正行為を目的に作られた既知のプロセスや,試験中に使用が許可されていないプロセス名に合致するプロセスの起動を検知したとき,管理者サーバに通知する.

検証実験では,不正プロセスの検出が実際に行われ,実用レベルまで不正プロセスを検出することができた.
また,試験者にPC上での行為が常時監視されているという圧力も,不正行為の抑止につながるため,システムが有用であると述べられていた.

\subsection{PC操作ログと映像ログを用いた業務行動モニタリングシステム}
監視システムにより作業者の行動をモニタリングすることを提案した研究がある\cite{Toba2010}.
この研究では,ビデオカメラで取得した映像と,PCの操作ログの分析結果を組み合わせることによって,業務動作の監視を行う.
構築したシステムによりデータを取得し,業務動作をモニタリング分析し,業務の効率化を図った.

作業者の映像は,業務が行われている場所に設置した複数の監視カメラによって取得する.
PCの操作ログは,事前に監視対象のPCにインストールしたアプリケーションから取得する.
作業者の業務行動は,映像から取得した情報と,PCの操作ログを統合して分類する.

検証実験では,実際の業務行動を監視し,行動分析を行った.
構築したシステムにより業務行動の検出が可能となり,作業者の行動監視に向け提案手法が有効である見通しが得られた.
また,PC操作ログだけでなく,映像ログの分析も用いることで,業務行動を解析できる範囲が大きく広がることがわかった.

\subsection{本研究の特徴}
本研究では,PCの監視システムをPCワークの進捗監視に利用した.
対象はPC上で行われる作業のみに絞り,内部のアプリケーションを監視する.

作業者の監視は,予め監視対象のPCにダウンロードしたアプリケーションにより行う.
アプリケーションの起動状況及び,PC上で表示している画面のキャプチャ画像を取得・分析することによって,作業者の動作を検出する.
検出した作業者の動作と,事前に設定したPCワークの作業予定の情報を組み合わせて,作業記録として出力する.
この作業記録を自動的に管理者や作業者に提出するシステムを構築し,作業者の生産性の向上を図る.

\section{論文構成}
本論文は5章で構成される.
2章では,PCワークにおける進捗監視システムの現状と,進捗監視を行うにあたり必要なプロセス監視方式について述べる.
3章では,提案・構築したシステムについて述べる.
4章では,システムの有用性を検証した実験について述べる.
5章では,本研究についてまとめ,今後の課題を述べる.
