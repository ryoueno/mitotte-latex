\chapter{進捗監視システム}

本章では,PCワークにおける進捗監視システムの現状と,進捗監視を行うにあたり必要なプロセス監視方式について述べる.

\section{現在の監視システムについて}

本節では,現在利用されている監視システムの例と,問題点について述べる.

\subsection{監視システムの例}
PCを利用したユーザの行動を監視するシステムの例として,オンライン試験が挙げられる\cite{Hari2006}.
インターネットの普及により,紙媒体による試験からPCによるオンライン試験へと移行が進んでいる.
オンライン試験は,試験監督の監視下にあるPCを用いて,不正行為が検出可能な状況で行われる必要がある.
そのため,受験者のPCに公正さを保つためのアプリケーションを導入することにより,オンライン試験が実現されている.

オンライン試験で利用されるアプリケーションでは,受験者の動作検出や回答状況の確認などを行うことができる.
受験者をリアルタイムで監視することにより,公正に試験を受けているという証明を行う.

\subsection{監視システムの問題点}
現状の監視システムにおいては,自動化が不十分であり,システム自体が監視対象とPCのみで完結させることはできていない.
先ほどの例では,不正行為の検知を監督者が目視で行うことになっている.
PC自体のプロセスを監視し,不正な動作を行っていないかどうかをプログラムによって判定するシステムも存在するが,100%不正行為を検知することはできない.

\section{監視方式}
PCの監視方式は大まかに,エージェントレス型と,エージェント型に分類される\cite{Hari2006}.
エージェントレス型は,特定のプログラムを監視対象のPC内で動作させることなく,不正プロセスなどを監視する方式である.
エージェント型は,特定のアプリケーションを対象のPCにインストールさせて監視する方式である.

\subsection{エージェントレス型}
エージェントレス型の監視方式には,以下のようなものがある.

\subsubsection{リモートログイン}
ユーザに監視対象のPCから管理サーバへアクセスさせることで,監視を行う方法である.
監視する領域が管理サーバのアプリケーション内に限定される.
また,リモートへアクセスするため,各ユーザのアカウントを事前に発行しておく必要がある.

\subsubsection{送受信パケット監視}
ネットワーク上を流れるパケットを監視することによって,不正な情報交換を監視する方法である.
監視対象が送受信パケットに限定されるため,外部にアクセスするようなプロセスしか監視できず,ネットワークを利用しないPC内に閉じたアプリケーションの動作は確認できない.

\subsection{エージェント型}
エージェント型の監視方式には,以下のようなものがある.

\subsubsection{インストール型}
監視対象のPCに監視用のアプリケーションを事前にインストールしておき,監視を行う方法である.
専用のデバイスではなく,各個人のPCを対象とする場合,対象者自身を監視するソフトウェアをインストールさせること自体が難しい場合もある.

\subsubsection{エージェント送り込み型}
Webブラウザなどを使用し,アプリケーションの追加機能としてダウンロードさせることで,監視させる方法である.
インストールの操作が不要であるため,インストール型に対して抵抗感は少なくなる.
