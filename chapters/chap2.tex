\chapter{進捗監視システム}

本章では,PCワークにおける進捗監視システムの現状と,進捗監視を行うにあたり必要なプロセス監視方式について述べる.

\section{現在の監視システムについて}

ここでは,現在利用されている監視システムの例と,問題点について述べる.

\subsection{監視システムの例}
PCを利用したユーザの行動を監視する場面の一例として,オンライン試験が挙げられる.インターネットの普及により,
紙媒体による試験からPCによるオンライン試験へと移行が進んでいる.オンライン試験は,試験監督の監視下にあるPCを用いて,
不正行為が検出可能な状況で行われる必要がある.そのため,受験者のPCに公正さを保つためのアプリケーションを導入することにより,
オンライン試験が実現されている.

オンライン試験で利用されるアプリケーションでは,受験者の動作検出や回答状況の確認などを行うことができる.
受験者をリアルタイムで監視しておくことにより,公正に試験を受けているという証明を行う.

\subsection{監視システムの問題点}
現状の監視システムにおいては,自動化が不十分であり,システム自体が監視対象とコンピュータのみで完結させることはできていない.
先ほどの例では,不正行為を検知を監督者が目視で行うことになっている.PC自体のプロセスを監視し,不正な動作を行っていないかどうかを
プログラムによって判定するシステムも存在するが,100%不正行為を検知することはできない.

\section{プロセス監視方式}

\subsection{エージェント型}

\subsection{エージェントレス型}


%以降削除すること
\clearpage
\noindent
一二三四五六七八九零一二三四五六七八九零一二三四五六七八九零一二三四五\\
二\\
三\\
四\\
五\\
六\\
七\\
八\\
九\\
零\\
一\\
二\\
三\\
四\\
五\\
六\\
七\\
八\\
九\\
零  行数と列数の設定テスト 30行×35文字 = 1050文字/ページ\\
一\\
二\\
三\\
四\\
五\\
六\\
七\\
八\\
九\\
零

