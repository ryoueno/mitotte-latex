現在,勉強やオフィスワークでもPCを始めとするデジタル機器を利用することが一般的となっている.
PCと人間が向き合い,個人あるいは組織における作業をする機会が多くなるにつれ,個人の作業を管理し生産性を高めていく必要が出てきた.
それを実現する方法として,個人の作業行動を監視し,進捗状況を自身及び第三者に提供することを考えた.
本研究では,PCワークの作業状況を記録・報告を自動化し,個人の生産性を向上させることを目的としたシステムを構築する.

関連研究では,PCの起動プロセスを監視し,オンラインテストにおける不正行為の防止する試みが行われている.
PCの起動プロセスを監視する方式として,特定のプログラムをクライアントPC内で動作させることなく監視を行うエージェントレス型と,特定のプログラムをクライアントPCにインストールし,動作させてを行うエージェント型に分けられる.本研究では,エージェント型を採用し,クライアントPC内でアプリケーションを動作させ,PCワークの進捗監視システムを実現する.

構築したシステムでは,アプリケーションの起動状況の記録及びPCの動作画面のスクリーンショットを解析した結果を用いて作業状況の記録を生成する.スクリーンショットの解析では,ユーザがPC上に表示している画面から,何のサービスを利用しているのかを検出し,記録する.ユーザは自身の作業スケジュールを管理することが可能であり,その作業スケジュールと,実際の作業記録を合わせたものを自動的に第三者に送信することにより,報告までの自動化が完成する.

システムの運用を行なった結果,ユーザが作成した予定を元に作業時間を記録し,自動的に報告を行うことができた.
ユーザが自身の作業を行うように促す効果があることも判明し,本システムの有用性が高いことが検証された.
