謝辞には,論文を書くにあたりお世話になった方々へ感謝の言葉を記述します.
実は論文内で非常に良く見られる項目でもあるため,漏れが無いように気をつける必要があります.

少なくとも,指導をおこなった教員,一緒に学んだり励まし合ったりした同じ研究室のメンバーに対する感謝の気持ちを書くことをおすすめします.

たとえ,あまり感謝していなかったとしても,礼儀として書いておいた方が良いでしょう.
論文は何十年も残るモノですから,誰に見られるかわからないということを想定して下さい.
また,何年か後には皆さんの気持ちも変化するものですから,あとで後悔しないように慎重に記述して下さい.

宮治の場合には,上記の他に,両親や研究の際に利用したフリーソフト(今でいうオープンソースのソフトウェア)の作者にも感謝の気持ちを述べました.