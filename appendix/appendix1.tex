\chapter{ユーザのアンケート結果}
作業を行う立場のユーザの視点から本システムを評価するためのアンケートは,監視対象であった9名ユーザを対象に実施した.
以下に,質問項目と回答内容を示す.

\subsubsection*{本システムを使うことによって,作業をやらないといけない気になりましたか?}

\begin{table}[htb]
  \begin{tabular}{ll}
    なった & 6人 \\
    少しなった & 3人 \\
    あまりなっていない & 0人 \\
    ならなかった & 0人
  \end{tabular}
\end{table}

\subsubsection*{本システムを使うことによって生産性が向上すると思いますか?}

\begin{table}[htb]
  \begin{tabular}{ll}
    とても思う & 1人 \\
    思う & 8人 \\
    あまり思わない & 0人 \\
    思わない & 0人
  \end{tabular}
\end{table}

\subsubsection*{自身の作業が報告されることに抵抗がありましたか?}

\begin{table}[htb]
  \begin{tabular}{ll}
    とても抵抗がある & 0人 \\
    抵抗がある & 7人 \\
    あまり抵抗がない & 2人 \\
    全く抵抗がない & 0人
  \end{tabular}
\end{table}

\subsubsection*{作業の際,本システムに監視されていることを意識しましたか?}

\begin{table}[htb]
  \begin{tabular}{ll}
    とても意識した & 2人 \\
    意識した & 5人 \\
    あまり意識していない & 2人 \\
    全く意識していない & 0人
  \end{tabular}
\end{table}

\subsubsection*{予定の設定はわかりやすかったですか?}

\begin{table}[htb]
  \begin{tabular}{ll}
    とてもわかりやすい & 0人 \\
    わかりやすい & 2人 \\
    わかりにくい & 7人 \\
    とてもわかりにくい & 0人
  \end{tabular}
\end{table}

\subsubsection*{予定を管理するにあたって,本システムは使いやすいと思いましたか?}

\begin{table}[htb]
  \begin{tabular}{ll}
    とても使いやすい & 0人 \\
    使いやすい & 1人 \\
    やや使いにくい & 8人 \\
    とても使いにくい& 0人
  \end{tabular}
\end{table}

\subsubsection*{本システムから作業開始時間の通知は,役に立ちましたか?}

\begin{table}[htb]
  \begin{tabular}{ll}
    とても役に立った & 0人 \\
    役に立った & 7人 \\
    あまり役に立たなかった & 2人 \\
    全く役に立たなかった & 0人
  \end{tabular}
\end{table}

\subsubsection*{本システムからの通知の頻度はどうでしたか?}

\begin{table}[htb]
  \begin{tabular}{ll}
    少ない & 0人 \\
    ちょうどいい & 3人 \\
    多い & 5人 \\
    とても多い & 1人
  \end{tabular}
\end{table}

\subsubsection*{本システムを利用して,役に立った部分を教えてください}

\begin{table}[htb]
  \begin{tabular}{p{14cm}} \hline
    論文が忘れられなくなった \\ \hline
    システム開発をやらなければならないと思った \\ \hline
    他のゼミ生がどれくらい卒論に取り組んでいるのかがわかって刺激になった \\ \hline
    グラフ化された実績が共有されることでやらなきゃいけない気持ちが大きくなった \\ \hline
    作業時間に対する達成感がモチベーションにつながった \\ \hline
    予定を設定し,通知してもらうことによって,やる気が向上した \\ \hline
    実作業時間が,管理者だけでなく自分も把握できる部分 \\ \hline
  \end{tabular}
\end{table}

\subsubsection*{本システムを利用して,不足している部分をお聞かせください}

\begin{table}[htb]
  \begin{tabular}{p{14cm}} \hline
    予定のとこのUI \\ \hline
    ネット環境の無いところで作業した分が反映されないところ \\ \hline
    プロジェクトの管理が自分の頭の中でカレンダーを想像しながらじゃないといけなくて大変だった \\ \hline
    オフライン環境での動作 \\ \hline
    予定の達成率などを,数字で示されるとより良いと感じた. \\ \hline
    予定の入力や変更の時のとっつきやすさ,手軽さが少し足りない.どこを触れば予定が編集できるかわかりにくいのと,
    簡単に変更ができないことから予定の編集を行うことが億劫になってしまう場合がある. \\ \hline
    スマホから予定を登録できるようになると良かったと思う.外出先などPCを使えない時,手軽に予定を変更したかった \\ \hline
  \end{tabular}
\end{table}

\clearpage

\subsubsection*{このようなシステムによって進捗管理を自動化することは現実的だと思いますか? またその理由をお聞かせください}

\begin{table}[htb]
  \begin{tabular}{p{14cm}} \hline
    現実的だとおもう.それぞれタスクが違うって考えるのが現実的で人間か管理するのには限界がある. \\ \hline
    現実的だと思う.人間ではなく,システムによる管理であればプライバシーが守られるから. \\ \hline
    現実的.ただし導入先による.メディアリテラシーのある人を対象に行ういいと考える. \\ \hline
    思わない.プライバシーの面で実現が難しそう. \\ \hline
    現実的だと思う.PC上で学習するものに関しては,自動で一括管理ができる上,今後そのような課題も増えると思うから. \\ \hline
    現実的だと思う.進捗管理において,監視や予定設定による被管理者の不快感が,このようなシステムの実現の障壁になると思う.
    しかしながら,このシステムは被管理者が自らの状態を自ら報告することが可能であり,自分の予定や調子に合わせて,不快感なく進捗管理を行ってもらうことが可能だと思ったため.\\ \hline
  \end{tabular}
\end{table}
