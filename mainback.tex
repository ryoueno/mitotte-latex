\documentclass[a4paper,11pt,oneside,openany]{jsbook}
%
\usepackage{fancyhdr}
\usepackage{amsmath,amssymb}
\usepackage{bm}
\usepackage[dvipdfmx]{graphicx}
\usepackage{subfigure}
\usepackage{url}
\usepackage{verbatim}
\usepackage{wrapfig}
\usepackage{ascmac}
\usepackage{fancyvrb}
\usepackage{makeidx}
\usepackage{comment}
\usepackage{float}

%%%%%%%%%%ここより上は必要なスタイルを読み込みたい時にのみ編集%%%%%%%%%%%%%
\usepackage{myjlab}
\makeindex
%%%%%%%%%% 念のため \usepackegeの一番下に myjlab があるように%%%%%%%%%%%%%

%%%%%%%%%%基本情報設定変更の必要なし%%%%%%%%%%%%%%%%%%%%%%%%%%%%
\def\daigaku{青山学院大学}
\def\gakubu{社会情報学部}
\def\gakka{社会情報学科}
\def\syubetsu{卒業論文}
\def\labname{宮治研究室}
\def\chiefexaminer{宮治 裕 准教授}

%%%%%%%%%%%%%%%%%%%%%%%%%%%%%%%%%%%%%%%
% ここから先「ここまで個人設定」の範囲に
% 各自の固有の情報を記入して下さい
%%%%%%%%%%%%%%%%%%%%%%%%%%%%%%%%%%%%%%%
\def\nendo{2013年度}
\def\teisyutsu{2014年~~1月}

\def\snum{15387019}
\def\jname{宮治 裕}
\def\thesistitle{宮治研における論文作成について} %タイトルを記入
\def\thesissubtitle{\LaTeX の利用} %サブタイトルを記入 ない場合はコメントアウトし,titlepage.tex を読み込むように変更


%%%%%%%%%% ここまで個人設定 %%%%%%%%%%%%%%


% 以下3行編集不要
\begin{document}
\linesparpage{30} %行数指定
\mojiparline{35} %文字数指定


%%%%%%%%%%%%%%%%%%%%%%%%%%%%%%%%%%%%%%%
% ここから先「ここまでタイトル表示設定」の範囲は
% サブタイトル有りの場合とサブタイトルなしの場合の
% 条件に応じてコメントアウト部分を変更して下さい
%%%%%%%%%%%%%%%%%%%%%%%%%%%%%%%%%%%%%%%
\begin{titlepage}
\begin{center}{\large
\nendo ~~\daigaku ~\gakubu ~~\syubetsu
}
\end{center}
\begin{flushright}

{\large
指導教員: \chiefexaminer \\ % 主査
%副査:□□□□ 教授              % 副査
}
\end{flushright}
\begin{center}
\vspace*{150truept}
{\huge \thesistitle}\\ % タイトル
\vspace{10truept}
{\Large ―~\thesissubtitle~―}\\ % サブタイトル(なければコメントアウト)
\vspace{150truept}
%\vspace{120truept}
{\Large
\begin{tabular}{rl}
%{\Large 学籍番号 \snum}\\ % 学籍番号
学籍番号 & \hspace{1zw}{\snum}\\
%{\Large \jname}\\ % 著者
 & \\
氏名 & \hspace{1zw}{\jname}\\
\end{tabular}
}\\
\vspace{70truept}
{\large \teisyutsu}\\ % 提出日
\end{center}
\end{titlepage}
%サブタイトル有りの場合
%\begin{titlepage}
\begin{center}{\large
\nendo ~~\daigaku ~\gakubu ~~\syubetsu
}
\end{center}
\begin{flushright}

{\large
指導教員: \chiefexaminer \\ % 主査
}
\end{flushright}

\begin{center}
\vspace*{150truept}
{\huge \thesistitle}\\ % タイトル
\vspace{170truept}
%\vspace{120truept}
{\Large
\begin{tabular}{rl}
%{\Large 学籍番号 \snum}\\ % 学籍番号
学籍番号 & \hspace{1zw}{\snum}\\
%{\Large \jname}\\ % 著者
 & \\
氏名 & \hspace{1zw}{\jname}\\
\end{tabular}
}\\
\vspace{70truept}
{\large \teisyutsu}\\ % 提出日
\end{center}
\end{titlepage}%サブタイトル無しの場合

%%%%%%% ここまでタイトル表示設定 %%%%%%%%%%


% ------------------- ここから先「ここまで共通」の範囲は編集不要 --------------------
\frontmatter
%%% 論文要旨
\chapter*{論文要旨}
%\thispagestyle{empty}
\addcontentsline{toc}{chapter}{論文要旨}
現在,勉強やオフィスワークでもPCを始めとするデジタル機器を利用することが一般的となっている.
PCと人間が向き合い,個人あるいは組織における作業をする機会が多くなるにつれ,個人の作業を管理し生産性を高めていく必要が出てきた.
それを実現する方法として,個人の作業行動を監視し,進捗状況を自身及び第三者に提供することを考えた.
本研究では,PCワークの作業状況を記録・報告を自動化し,個人の生産性を向上させることを目的としたシステムを構築する.

% abstract.texの中は \chapterなど書かずに単なるテキストを入力する

%%% 謝辞
\chapter*{謝辞}
%\thispagestyle{empty}
\addcontentsline{toc}{chapter}{謝辞}
 この研究を遂行するにあたり,終始暖かく見守って下さった宮治裕准教授に深く感謝いたします.
共に頑張ってきた宮治研究室の皆さまには,多くの刺激を受け,力をもらいました.
本当にありがとうございました.

宮治研究室の一員として,本研究開発に取り組むことができ,幸せでした.
心より感謝いたします.

% thanks.texの中は \chapterなど書かずに単なるテキストを入力する

%%% 目次
\tableofcontents
\mainmatter
% ------------------- ここまで共通 -------------------


%%%%%%%%%%%%%%%%%%%%%%%%%%%%%%%%%%%%%%%
% ここから先「ここまで論文本文」の範囲を
% 各自の章立てや付録にあわせて編集して下さい
%%%%%%%%%%%%%%%%%%%%%%%%%%%%%%%%%%%%%%%

%%% 本文ここから chap1 chap同様に必要なだけ章を入れる
\chapter{はじめに}
本論文では,PCワークに限り,進捗管理と行動監視を行い,目標達成を支援するシステムについて記述する.

まず,本研究をおこなう背景となった事柄について述べる.
次に,研究目的の詳細を記述した後,類似研究との相違や関連研究とのつながりについて解説する.
また,次章以降の本論文の構成についてその概略を述べる.

\section{背景}
現在,勉強やオフィスワークでもPCを始めとするデジタル機器を利用することが一般的となっている.
それにより,今まで一般的であったオフィスなど,ひとつの拠点に集まって仕事をするのではなく,ひとりひとりが拠点を離れて仕事をすることも可能となった.在宅勤務は,育児や介護などのために自宅を離れられない個人にとって,家庭生活と仕事を両立するための手段として期待されている.場所の制約をなくすことは,出勤時間そのものも削減することができ,効率的に時間を使うことができる利点がある.

%以降削除すること
\clearpage
\noindent
一二三四五六七八九零一二三四五六七八九零一二三四五六七八九零一二三四五\\
二\\
三\\
四\\
五\\
六\\
七\\
八\\
九\\
零\\
一\\
二\\
三\\
四\\
五\\
六\\
七\\
八\\
九\\
零  行数と列数の設定テスト 30行×35文字 = 1050文字/ページ\\
一\\
二\\
三\\
四\\
五\\
六\\
七\\
八\\
九\\
零

 % 1章
%\chapter{進捗監視システム}

本章では,PCワークにおける進捗監視システムの現状と,進捗監視を行うにあたり必要なプロセス監視方式について述べる.

\section{現在の監視システムについて}

ここでは,現在利用されている監視システムの例と,問題点について述べる.

\subsection{監視システムの例}
PCを利用したユーザの行動を監視する場面の一例として,オンライン試験が挙げられる.
インターネットの普及により,紙媒体による試験からPCによるオンライン試験へと移行が進んでいる.
オンライン試験は,試験監督の監視下にあるPCを用いて,不正行為が検出可能な状況で行われる必要がある.
そのため,受験者のPCに公正さを保つためのアプリケーションを導入することにより,オンライン試験が実現されている.

オンライン試験で利用されるアプリケーションでは,受験者の動作検出や回答状況の確認などを行うことができる.
受験者をリアルタイムで監視しておくことにより,公正に試験を受けているという証明を行う.

\subsection{監視システムの問題点}
現状の監視システムにおいては,自動化が不十分であり,システム自体が監視対象とコンピュータのみで完結させることはできていない.
先ほどの例では,不正行為を検知を監督者が目視で行うことになっている.
PC自体のプロセスを監視し,不正な動作を行っていないかどうかをプログラムによって判定するシステムも存在するが,100%不正行為を検知することはできない.

\section{監視方式}
PCの監視方式は大まかに,エージェントレス型と,エージェントレス型に分類される.
エージェントレス型は,特定のプログラムを監視対象のPC内で動作させることなく不正プロセスなどを監視する方式である.
エージェントレス型は,特定のアプリケーションを対象のPCにインストールさせて監視する方式である.

\subsection{エージェントレス型}
エージェントレス型の監視方式には,以下のようなものがある.

\subsubsection{リモートログイン}
ユーザに監視対象のPCから管理サーバへアクセスさせることで,監視を行う方法である.
監視される領域が管理サーバのアプリケーション内に限定される.
また,事前に各ユーザのアカウントを発行しておく必要がある.

\subsubsection{送受信パケット監視}
ネットワーク上を流れるパケットを監視することによって,不正な情報交換を監視する方法である.
監視対象が送受信パケットに限定されるため,外部にアクセスするようなプロセスしか監視できず,ネットワークを利用しないPC内に閉じたアプリケーションの動作は確認できない.

\subsection{エージェント型}
エージェント型の監視方式には,以下のようなものがある.

\subsubsection{インストール型}
監視対象のPCに監視用のアプリケーションを事前にインストールしておき,監視を行う方法である.
専用のデバイスではなく,各個人のPCを対象とする場合,自信を監視するソフトウェアをインストールさせること自体が難しい場合もある.

\subsubsection{エージェント送り込み型}
Webブラウザなどを使用し,アプリケーションの追加機能としてダウンロードさせることで,監視させる方法である.
インストールの操作が不要であるため,インストール型に対して抵抗感は少なくなる.

%以降削除すること
\clearpage
\noindent
一二三四五六七八九零一二三四五六七八九零一二三四五六七八九零一二三四五\\
二\\
三\\
四\\
五\\
六\\
七\\
八\\
九\\
零\\
一\\
二\\
三\\
四\\
五\\
六\\
七\\
八\\
九\\
零  行数と列数の設定テスト 30行×35文字 = 1050文字/ページ\\
一\\
二\\
三\\
四\\
五\\
六\\
七\\
八\\
九\\
零

 % 2章
\chapter{システム構成}

\section{システム概要}

\section{予定管理機能}

\section{アプリケーションによる監視}

\subsection{起動状況の記録}

\subsection{サービス検出}

\subsection{ユーザへの通知}

\section{作業記録の生成}

%以降削除すること
\clearpage
\noindent
一二三四五六七八九零一二三四五六七八九零一二三四五六七八九零一二三四五\\
二\\
三\\
四\\
五\\
六\\
七\\
八\\
九\\
零\\
一\\
二\\
三\\
四\\
五\\
六\\
七\\
八\\
九\\
零  行数と列数の設定テスト 30行×35文字 = 1050文字/ページ\\
一\\
二\\
三\\
四\\
五\\
六\\
七\\
八\\
九\\
零

 % 3章
%\chapter{実験}

\section{目的}

\section{対象}

\section{前提条件}

\subsection{達成目標}
ユーザの状況,達成目標,具体的に何の作業をやるのか

\subsection{作業の定義}
使用するアプリケーション,当実験に限定している部分を記述

\section{方法}

\section{結果}

\section{考察}

%以降削除すること
\clearpage
\noindent
一二三四五六七八九零一二三四五六七八九零一二三四五六七八九零一二三四五\\
二\\
三\\
四\\
五\\
六\\
七\\
八\\
九\\
零\\
一\\
二\\
三\\
四\\
五\\
六\\
七\\
八\\
九\\
零  行数と列数の設定テスト 30行×35文字 = 1050文字/ページ\\
一\\
二\\
三\\
四\\
五\\
六\\
七\\
八\\
九\\
零

 % 4章
%\chapter{おわりに}

\section{まとめ}

\section{今後の課題}

%以降削除すること
\clearpage
\noindent
一二三四五六七八九零一二三四五六七八九零一二三四五六七八九零一二三四五\\
二\\
三\\
四\\
五\\
六\\
七\\
八\\
九\\
零\\
一\\
二\\
三\\
四\\
五\\
六\\
七\\
八\\
九\\
零  行数と列数の設定テスト 30行×35文字 = 1050文字/ページ\\
一\\
二\\
三\\
四\\
五\\
六\\
七\\
八\\
九\\
零

 % 5章
%\include{chap6} % 6章

%%% 付録 -- 必要なければ以下を2行コメントアウト
\appendix
\chapter{プログラムの動作方法}
本研究にて用いたプログラムについて解説する.

\section{ファイル構成}
プログラムのフォルダ内は,主に4つのファイルから構成される.

ああああいいいい

ううううええええ

これらを○○に設置し,以下の手順にそって起動する.

\section{起動方法}
まず,ウェブサーバを動かした状態にし,外部クライアント(Webブラウザから),以下のURLにアクセスする.


\section{表示の見方}
実験に利用するための,実行結果は test.log ファイルに出力されている.

このファイルは4つのカラムからなる CSV形式のファイルである.
第1列には,…





%\include{appendixB} %必要に応じて付録の数を増やす

\clearpage
%%%%%%%%%% ここまで論文本文 %%%%%%%%%%%%%%


% ************** ここから先の範囲は編集不要 ****************
%%% 参考文献
\bibliographystyle{junsrt}
\bibliography{myrefs}
% myrefs.bib の中はサンプルファイルを参考に記述

\newpage
\printindex
%
\end{document}
