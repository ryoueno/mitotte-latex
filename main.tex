%%%%%%%%%%%%%%%%%%%%%%%%%%%%%%%%%%%%%%%
% プリアンブル 各種設定情報
%%%%%%%%%%%%%%%%%%%%%%%%%%%%%%%%%%%%%%%
\documentclass[a4paper,11pt,oneside,openany]{jsbook}
%
\usepackage{fancyhdr}
\usepackage{amsmath,amssymb}
\usepackage{bm}
\usepackage[dvipdfmx]{graphicx}
\usepackage{subfigure}
\usepackage{url}
\usepackage{verbatim}
\usepackage{wrapfig}
\usepackage{ascmac}
\usepackage{fancyvrb}
\usepackage{makeidx}
\usepackage{comment}
\usepackage{eclbkbox}
\usepackage{float}

%%%%%%%%%%ここより上は必要なスタイルを読み込みたい時にのみ編集%%%%%%%%%%%%%
\usepackage{myjlab}
\makeindex
%%%%%%%%%% 念のため \usepackegeの一番下に myjlab があるように%%%%%%%%%%%%%

%%%%%%%%%%基本情報設定変更の必要なし%%%%%%%%%%%%%%%%%%%%%%%%%%%%
\daigaku{青山学院大学}
\gakubu{社会情報学部}
\gakka{社会情報学科}
\syubetsu{卒業論文}
\labname{宮治研究室}
\chiefexaminer{宮治 裕 准教授}

%%%%%%%%%%%%%%%%%%%%%%%%%%%%%%%%%%%%%%%
% ここから先「ここまで個人設定」の範囲に
% 各自の固有の情報を記入して下さい
%%%%%%%%%%%%%%%%%%%%%%%%%%%%%%%%%%%%%%%
\nendo{2017年度}
\teisyutsu{2017年~~1月}
\snum{18114035}
\jname{上野 涼}
\thesistitle{PCワークの進捗監視システム} %タイトルを記入
%\thesissubtitle{\LaTeX の利用} %サブタイトルを記入 ない場合はコメントアウト
%\SUBTtrue %サブタイトル有りの場合 ない場合は,コメントアウト
\SUBTfalse %サブタイトル無しの場合 有る場合は,コメントアウト
%%%%%%%%%% ここまで個人設定 %%%%%%%%%%%%%%

%%%%%%%%%%%%%%%%%%%%%%%%%%%%%%%%%%%%%%%
% ここから先,論文内原稿
% 「ここまで共通」まで編集不要
%%%%%%%%%%%%%%%%%%%%%%%%%%%%%%%%%%%%%%%
\begin{document}
\linesparpage{30} %行数指定
\mojiparline{35} %文字数指定
\pagestyle{empty}
\maketitle

\frontmatter
%%% 論文要旨
\chapter*{論文要旨}
%\thispagestyle{empty}
\addcontentsline{toc}{chapter}{論文要旨}
現在,勉強やオフィスワークでもPCを始めとするデジタル機器を利用することが一般的となっている.
PCと人間が向き合い,個人あるいは組織における作業をする機会が多くなるにつれ,個人の作業を管理し生産性を高めていく必要が出てきた.
それを実現する方法として,個人の作業行動を監視し,進捗状況を自身及び第三者に提供することを考えた.
本研究では,PCワークの作業状況を記録・報告を自動化し,個人の生産性を向上させることを目的としたシステムを構築する.

関連研究では,PCの起動プロセスを監視し,オンラインテストにおける不正行為の防止する試みが行われている.
PCの起動プロセスを監視する方式として,特定のプログラムをクライアントPC内で動作させることなく監視を行うエージェントレス型と,特定のプログラムをクライアントPCにインストールし,動作させてを行うエージェント型に分けられる.本研究では,エージェント型を採用し,クライアントPC内でアプリケーションを動作させ,PCワークの進捗監視システムを実現する.

構築したシステムでは,アプリケーションの起動状況の記録及びPCの動作画面のスクリーンショットを解析した結果を用いて作業状況の記録を生成する.スクリーンショットの解析では,ユーザがPC上に表示している画面から,何のサービスを利用しているのかを検出し,記録する.ユーザは自身の作業スケジュールを管理することが可能であり,その作業スケジュールと,実際の作業記録を合わせたものを自動的に第三者に送信することにより,報告までの自動化が完成する.

[wip] 結果はまだわかりません.結果はまだわかりません.結果はまだわかりません.結果はまだわかりません.結果はまだわかりません.

\pagestyle{plain}
\pagenumbering{roman}
% abstract.texの中は \chapterなど書かずに単なるテキストを入力する

%%% 謝辞
\chapter*{謝辞}
%\thispagestyle{empty}
\addcontentsline{toc}{chapter}{謝辞}
謝辞には,論文を書くにあたりお世話になった方々へ感謝の言葉を記述します.
実は論文内で非常に良く見られる項目でもあるため,漏れが無いように気をつける必要があります.

少なくとも,指導をおこなった教員,一緒に学んだり励まし合ったりした同じ研究室のメンバーに対する感謝の気持ちを書くことをおすすめします.

たとえ,あまり感謝していなかったとしても,礼儀として書いておいた方が良いでしょう.
論文は何十年も残るモノですから,誰に見られるかわからないということを想定して下さい.
また,何年か後には皆さんの気持ちも変化するものですから,あとで後悔しないように慎重に記述して下さい.

宮治の場合には,上記の他に,両親や研究の際に利用したフリーソフト(今でいうオープンソースのソフトウェア)の作者にも感謝の気持ちを述べました.
% thanks.texの中は \chapterなど書かずに単なるテキストを入力する

%%% 目次
\tableofcontents
% 目次は自動生成される
%
\mainmatter
\pagestyle{fancy}
\pagenumbering{arabic}
%%%%%%%%%% 「ここまで共通」 %%%%%%%%%%%%%%


%%%%%%%%%%%%%%%%%%%%%%%%%%%%%%%%%%%%%%%
% ここから先「ここまで論文本文」の範囲を
% 各自の章立てや付録にあわせて編集して下さい
%%%%%%%%%%%%%%%%%%%%%%%%%%%%%%%%%%%%%%%

%%% 本文ここから chap1 chap2 chap3 同様に必要なだけ章を入れる
\chapter{はじめに}
本論文では,PCワークに限り,進捗管理と行動監視を行い,目標達成を支援するシステムについて記述する.

まず,本研究をおこなう背景となった事柄について述べる.
次に,研究目的の詳細を記述した後,類似研究との相違や関連研究とのつながりについて解説する.
また,次章以降の本論文の構成についてその概略を述べる.

\section{背景}
現在,勉強やオフィスワークでもPCを始めとするデジタル機器を利用することが一般的となっている.
それにより,今まで一般的であったオフィスなど,ひとつの拠点に集まって仕事をするのではなく,ひとりひとりが拠点を離れて仕事をすることも可能となった.在宅勤務は,育児や介護などのために自宅を離れられない個人にとって,家庭生活と仕事を両立するための手段として期待されている.場所の制約をなくすことは,出勤時間そのものも削減することができ,効率的に時間を使うことができる利点がある.

一方、そのスケジュール管理が自分自身に任されていることがデメリットになる場合もある。仕事量を計算し、自己管理できなければ、リモートワークは成立できない。労働時間についても程度勤務者の裁量にゆだねられる。そのため企業にとっては労働時間の管理が難しくなっている。

%以降削除すること
\clearpage
\noindent
一二三四五六七八九零一二三四五六七八九零一二三四五六七八九零一二三四五\\
二\\
三\\
四\\
五\\
六\\
七\\
八\\
九\\
零\\
一\\
二\\
三\\
四\\
五\\
六\\
七\\
八\\
九\\
零  行数と列数の設定テスト 30行×35文字 = 1050文字/ページ\\
一\\
二\\
三\\
四\\
五\\
六\\
七\\
八\\
九\\
零

 % 1章

%%% 付録 -- 必要なければ以下を2行コメントアウト
\appendix
\chapter{プログラムの動作方法}
本研究にて用いたプログラムについて解説する.

\section{ファイル構成}
プログラムのフォルダ内は,主に4つのファイルから構成される.

ああああいいいい

ううううええええ

これらを○○に設置し,以下の手順にそって起動する.

\section{起動方法}
まず,ウェブサーバを動かした状態にし,外部クライアント(Webブラウザから),以下のURLにアクセスする.


\section{表示の見方}
実験に利用するための,実行結果は test.log ファイルに出力されている.

このファイルは4つのカラムからなる CSV形式のファイルである.
第1列には,…





%\include{appendixB} %必要に応じて付録の数を増やす

%\clearpage
%%%%%%%%%% ここまで論文本文 %%%%%%%%%%%%%%


% ************** ここから先の範囲は編集不要 ****************
%%% 参考文献
\bibliographystyle{junsrt}
\bibliography{myrefs}
% myrefs.bib の中はサンプルファイルを参考に記述

\newpage
\printindex
%
\end{document}
