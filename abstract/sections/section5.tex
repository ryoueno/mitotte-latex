\section{まとめと今後の課題}
本研究では,個人のPCワークの進捗管理により,生産性を向上させることを目標としたシステムを構築した.
実験では実際にシステムの運用を行い,利用者からの評価を受け,提案したシステムの有用性を測った.

システムの運用を行なった結果,ユーザが作成した予定を元に作業時間を記録し,自動的に報告を行うことができた.
これにより,ユーザが自身の作業を行うように促す効果があることが判明した.
また,報告される情報も,監督者も作業者の現状を確認し,指導に役立つものであると評価された.
以上のことより,本システムの有用性が高いことが検証された.

本研究で構築したシステムの問題点は,個人の予定を正しくシステムに反映させることが難しいことである.

システムでユーザの正確な予定管理を実現することができなかった原因として,予定管理の入力に手間がかかることが挙げられる.
本研究では,構築したアプリケーション内で各個人の予定の入力がそれぞれで行われた.
この方法はユーザがシステムを能動的に利用し,個人が自ら予定を管理しなければならない.
ユーザが予定管理を自身で行うことができなかった場合,監督者も作業者の現状を正しく認識することができない.
管理される予定の精度を高めるためには,予定の入力によるユーザの負担を軽減し,手軽に利用できるシステムにしなければならない.
あるいは,監督者が作業内容や作業時間を見積もり,ユーザはそれを各自の予定に割り振るような形式にすることで,より確実に予定の管理が行われると考える.

進捗監視システムをより現実的なものにするためには,より正確な予定の管理機能を実現することが重要である.
そのためには,信頼性が高く,ユーザにとってより使いやすいシステムにしていく必要がある.
また,組織で運用していく中で,システムを通じてコミュニケーションがとれるサービスとして進化させていくことにより,組織全体の生産性を向上させるものになると考える.
