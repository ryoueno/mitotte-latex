\section{まとめと今後の課題}
本研究では,個人のPCワークの進捗管理により,生産性を向上させることを目標としたシステムを構築した.
実験では実際にシステムの運用を行い,利用者からの評価を受け,提案したシステムの有用性を測った.

システムの運用を行なった結果,ユーザが作成した予定を元に作業時間を記録し,自動的に報告を行うことができた.
これにより,ユーザが自身の作業を行うように促す効果があることが判明した.
また,報告される情報も,監督者も作業者の現状を確認し,指導に役立つものであると評価された.
以上のことより,本システムの有用性が高いことが検証された.

進捗監視システムをより現実的なものにするためには,より正確な予定の管理機能を実現することが重要である.
そのためには,信頼性が高く,ユーザにとってより使いやすいシステムにしていく必要がある.
また,組織で運用していく中で,システムを通じてコミュニケーションがとれるサービスとして進化させていくことにより,組織全体の生産性を向上させるものになると考える.
