\section{背景}
現在,勉強やオフィスワークでもPCを始めとするデジタル機器を利用することが一般的となっている.
それにより,今まで一般的であったオフィスなど,ひとつの拠点に集まって仕事をするのではなく,ひとりひとりが拠点を離れて仕事をすることも可能となった.
リモートワークは,育児や介護などのために自宅を離れられない個人にとって,家庭生活と仕事を両立するための手段として期待されている.
場所の制約をなくすことは,出勤時間そのものも削減することができ,効率的に時間を使うことができる利点がある\cite{tele2017}.

一方,リモートワークにおいて,スケジュール管理が自分自身に任されていることがデメリットになる場合もある.
仕事量を計算し,自己管理できなければ,リモートワークは成立できない.
労働時間に関しても,ある程度は勤務者の裁量にゆだねられる.
そのため企業にとっては労働時間の管理が難しくなっている\cite{Adachi2010}.

以上のことから,個人の遠隔地における作業を適切に管理し,生産性を維持していく必要があると考えた.
