\section{実験}
本章では,提案するシステムの有用性を検証する実験について述べる.

\subsection{目的}
この実験の目的は,ユーザと管理者両方の視点から有用性を測る.
まず,ユーザがシステムを利用することによって,作業の生産性が向上するかどうかを検証する.
また,管理者は,自動的に報告されるユーザの作業状況を確認し,現状を把握できるかどうかを確認する.

\subsection{実験手順}
監視対象であるユーザは,論文を執筆する際にアプリケーションを起動する.
本システムは,起動中のアプリケーションから収集される情報をもとに,作業記録の生成を行う.
生成された作業記録は,次の日の朝に,チャットサービスを通じて管理者に送信される.
それらの一連の流れを確認することで,システム自体の信頼性や情報の適正性を測る.

本システムに対するユーザ及び監督者の評価は,アンケート調査によって行う.
このアンケートでは,構築した監視システムの有用性や,実用的であるかどうかを調査する.

\subsection{結果}
進捗監視を行うにあたって,実際にユーザが作成した予定を元に作業時間を記録し,自動的に報告を行うことができた.
これにより,ユーザが作業を行う動機付けがなされ,行動を促すことができた.
また,報告される情報により,監督者の立場から,より具体的に作業者に指導を行うことが可能となった.
以上のことより,本システムの有用性が高いことが検証された.
