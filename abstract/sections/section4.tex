\section{実験}

この実験の目的は,ユーザと管理者両方の視点から有用性を測ることである.
まず,ユーザがシステムを利用することによって,作業の生産性が向上するかどうかを検証する.
また,管理者は,自動的に報告されるユーザの作業状況を確認し,現状を把握できるかどうかを確認する.

今回の実験では,研究室に所属する大学生の論文の執筆を監視する.
監視対象であるユーザは,論文を執筆する際にアプリケーションを起動する.
本システムは,起動中のアプリケーションから収集される情報をもとに,作業記録の生成を行う.
生成された作業記録は,次の日の朝に,チャットサービスを通じて管理者に送信される.
それらの一連の流れを確認することで,システム自体の信頼性や情報の適正性を測る.

システムの運用を行なった結果,ユーザが作成した予定を元に作業時間を自動的に記録し,チャットサービスを通じて報告を行うことができた.
サービス・アプリケーションの検出は,運用中に970回作動し,AmazonやYahooJapanなどのWebサービスや,SkypeやSlackなど,PC上で動作するアプリケーションが,作業に不要なものであるとして検出された.

本システムに対するユーザからの評価では,本システムを利用することによって,「やらないといけないという気持ちになった」という意見が複数得られた.
また,実績が可視化されることがモチベーションにつながるなど,報告される情報そのものに価値を感じるという意見も見られた.
これらの評価から,本システムによって作業を行う動機付けをすることは可能であり,行動を促すことを期待できることが分かった.

本研究で構築したシステムの問題点として,予定の入力に手間がかかることが挙げられる.
実験では,構築したアプリケーション内で各個人の予定の入力がそれぞれで行われた.
この方法はユーザがシステムを能動的に利用し,個人が自ら予定を管理しなければならない.
ユーザが予定管理を自身で行うことができなかった場合,監督者も作業者の現状を正しく認識することができない.
この問題を解決するためには,予定の入力におけるユーザの負担を軽減し,手軽に利用できるシステムにしていく必要がある.
あるいは,監督者が作業内容や作業儒医感を見積もり,ユーザはそれを各自の予定に割り振るような形式にすることで,より確実に予定の管理が行われると考える.
