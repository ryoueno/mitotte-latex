\chapter{はじめに}
本論文では,○○○を△△△することにより,□□を明らかとする研究について記述する.

まず,本研究をおこなう背景となった事柄について述べる.
次に,研究目的の詳細を記述した後,類似研究との相違や関連研究とのつながりについて解説する.
また,次章以降の本論文の構成についてその概略を述べる.

\section{背景}
研究の目的につながる背景事項を説明する.
その説明には,参考文献やデータを参照するように.

あまり詳しく書きすぎると,2章や3章などで書く内容が無くなったり重複したりしてしまうので,研究の目的の妥当性につながる程度の内容(詳細さ)でかまわない.

\section{研究目的}
背景によって,研究の大きな目的が導かれる.
その大きな目的を正確に定義した後,本研究にて実際にターゲットとする目的を詳細に記述する\footnote{大きな目的は1年間の研究ではカバーしきれない為}.

また,背景にて実際の詳細なターゲットの必要性を示した場合には,それの詳細な条件を記載する.

\section{関連研究}
類似研究(同じような研究)とは,どこが違うのか(ターゲット,手法,想定結果など)を述べる必要がある.
また,参考にする先行研究(他組織の研究でも良い)とどのような関連性があるのかを述べる.

場合によっては,関連研究が研究目的より先に書いてあった方が「ながれ」が良い場合もある.
また,関連研究を背景の中に入れてしまった方が良いケースもある.

\section{論文構成}
2章以降のざっくりとした流れを説明する.例えば

2章では,本研究にて活用した技術や関連サービスについてについて解説する.
3章では,提案・構築したシステムについて詳説する.
4章では,システムの有用性を検証する為に行った実験について記述する.
最後に5章において,本研究についてまとめ,今後の課題について述べる.

%以降削除すること
\clearpage
\noindent
一二三四五六七八九零一二三四五六七八九零一二三四五六七八九零一二三四五\\
二\\
三\\
四\\
五\\
六\\
七\\
八\\
九\\
零\\
一\\
二\\
三\\
四\\
五\\
六\\
七\\
八\\
九\\
零  行数と列数の設定テスト 30行×35文字 = 1050文字/ページ\\
一\\
二\\
三\\
四\\
五\\
六\\
七\\
八\\
九\\
零

