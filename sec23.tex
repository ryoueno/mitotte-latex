\section{main.tex: 大元となり,各自の情報,論文構成を記述}
main.tex では,各自の個人情報や,論文の章立ておよび章を構成するファイルの読み込みを設定する.

\subsection{各自の情報設定}
各自の情報を設定する際には,サブタイトルの有り/無しで設定事項が異なることに注意をする必要がある.
それぞれの方法について以下に記述する.
また,これらの作業が終わった時点で,本配布スタイルパッケージの動作確認をすることをおすすめする.

\subsubsection{サブタイトル有りの場合}
配布したファイルは,サブタイトルがある場合のサンプルになっている.
各自の 年度,提出年月,学籍番号,氏名,タイトル,サブタイトルを所定の命令内に記入する.
\begin{breakbox}
{\small
%footnotesize
\begin{verbatim}
\nendo{2013年度}
\teisyutsu{2014年~~1月}
\snum{15387019}
\jname{宮治 裕}
\thesistitle{宮治研における論文作成について} %タイトルを記入
\thesissubtitle{\LaTeX の利用} %サブタイトルを記入 ない場合はコメントアウト
\SUBTtrue %サブタイトル有りの場合 ない場合は,コメントアウト
%\SUBTfalse %サブタイトル無しの場合 有る場合は,コメントアウト
\end{verbatim}
}
\end{breakbox}

\subsubsection{サブタイトル無しの場合}
サブタイトル有りの場合と比較して2箇所の変更が必要である.
サブタイトルを記入する命令の先頭部分に \% 記号を入れ,コメントアウト状態にする.

\begin{breakbox}
{\small
\begin{verbatim}
%\thesissubtitle{\LaTeX の利用} %サブタイトルを記入 無い場合は,コメントアウト
\end{verbatim}
}
\end{breakbox}
もう一つは,その直下の2行
\begin{breakbox}
{\small
\begin{verbatim}
\SUBTtrue %サブタイトル有りの場合 無い場合は,コメントアウト
%\SUBTfalse %サブタイトル無しの場合 有る場合は,コメントアウト
\end{verbatim}
}
\end{breakbox}
以下の様に変更する.
\begin{breakbox}
{\small
\begin{verbatim}
%\SUBTtrue %サブタイトル有りの場合 無い場合は,コメントアウト
\SUBTfalse %サブタイトル無しの場合 有る場合は,コメントアウト
\end{verbatim}
}
\end{breakbox}

以上の設定で,表紙と各ページのヘッダ・フッタの情報が自動的に設定され,書式が整えられる.
\begin{boxnote}
\LaTeX では 「\verb+%+」はコメントを意味し,この記号から改行コードまでをコメントアウト状態として処理する.
\end{boxnote}
であることに注意すること.

\subsubsection{スタイルパッケージの動作確認}
サブタイトルの有り/無しに応じて適切に設定ができた段階で,一度各自の環境下でスタイルパッケージが正常動作することを確認して欲しい.
正常動作した場合には,本ファイルとほぼ同様の中身で,表紙と各ページのヘッダとフッタが各自の設定した情報が記載されたPDFファイルが出来上がるはずである.

まず,Macintoshの場合について記す.
各自のホームディレクトリ中のDropboxフォルダ内に,本スタイルパッケージが展開されている場合を前提として記述する.
\begin{enumerate}
\item まず,ターミナルを開く
\item 以下のコマンドを入力し,スタイルパッケージのあるフォルダに移動
\footnote{ここで \verb+$+記号は,コマンドプロンプトを表すため,入力しないように.}
\begin{screen}
{\small
\begin{verbatim}
 $ cd ~/Dropbox/Thesis
\end{verbatim}
}
\end{screen}

\item そこで,バッチファイル \verb+mklatex.bat+ を実行
\begin{screen}
{\small
\begin{verbatim}
 $ ./mklatex.bat
\end{verbatim}
}
\end{screen}

\item main.pdfファイルが作成され,プレビュー画面が自動で表示される
\item[\textbf{注}] mklatex.bat が実行できないというようなエラーが出た場合には,最初の一回だけ(次回から不要)以下の命令を入力する
\begin{screen}
{\small
\begin{verbatim}
 $ chmod 755 ./mklatex.bat
\end{verbatim}
}
\end{screen}
\end{enumerate}

正常動作しなかった場合には,出来上がった main.log ファイルを宮治に送付して欲しい.

Windowsの場合には,コマンドプロンプトを開き,目的のフォルダに移動し,バッチファイル(winmklatex.bat)を起動する.
\begin{screen}
{\small
\begin{verbatim}
 $ cd c:\My Documents\Dropbox\Thesis
 $ winmklatex.bat
\end{verbatim}
}
\end{screen}
main.pdfファイルができるので,エクスプローラからファイルをダブルクリックしてAcrobat Reader にて確認して欲しい.

\subsection{論文構成の設定}
main.texファイル内の以下の部分で論文構成を決定する.
一つの tex ファイルで論文を書ききることも可能だが,論文の構成や見通しが悪くなるために,このスタイルパッケージでは,main.tex ファイルから複数のtexファイルを読み込むようにしている.
「論文要旨」「謝辞」「論文の各章」「付録」などが,読み込まれるファイルである.

\subsubsection{論文要旨の読み込み}
まず,論文要旨は以下の形で定義されている.
\begin{breakbox}
{\small
\begin{verbatim}
\chapter*{論文要旨}
\addcontentsline{toc}{chapter}{論文要旨}
現在,勉強やオフィスワークでもPCを始めとするデジタル機器を利用することが一般的となっている.
PCと人間が向き合い,個人あるいは組織における作業をする機会が多くなるにつれ,個人の作業を管理し生産性を高めていく必要が出てきた.
それを実現する方法として,個人の作業行動を監視し,進捗状況を自身及び第三者に提供することを考えた.
本研究では,PCワークの作業状況を記録・報告を自動化し,個人の生産性を向上させることを目的としたシステムを構築する.

% abstract.texの中は \chapterなど書かずに単なるテキストを入力する
\end{verbatim}
}
\end{breakbox}
具体的には,\verb+現在,勉強やオフィスワークでもPCを始めとするデジタル機器を利用することが一般的となっている.
PCと人間が向き合い,個人あるいは組織における作業をする機会が多くなるにつれ,個人の作業を管理し生産性を高めていく必要が出てきた.
それを実現する方法として,個人の作業行動を監視し,進捗状況を自身及び第三者に提供することを考えた.
本研究では,PCワークの作業状況を記録・報告を自動化し,個人の生産性を向上させることを目的としたシステムを構築する.
+となっている部分で,abstract.texファイルを読み込んでいる.
コメントにも書いてあるように,abstract.tex 内には, \verb+\chapter+命令を入れない.

\subsubsection{謝辞の読み込み}
次に謝辞は以下の様に定義されている.
\begin{breakbox}
{\small
\begin{verbatim}
\chapter*{謝辞}
\addcontentsline{toc}{chapter}{謝辞}
 この研究を遂行するにあたり,終始暖かく見守って下さった宮治裕准教授に深く感謝いたします.
共に頑張ってきた宮治研究室の皆さまには,多くの刺激を受け,力をもらいました.
本当にありがとうございました.

宮治研究室の一員として,本研究開発に取り組むことができ,幸せでした.
心より感謝いたします.

% thanks.texの中は \chapterなど書かずに単なるテキストを入力する
\end{verbatim}
}
\end{breakbox}
論文要旨と同様に thanks.tex ファイルに \verb+\chapter+命令を入れずに記述する.

\subsubsection{目次の設定}
次に目次が定義されている.
\begin{breakbox}
{\small
%footnotesize
\begin{verbatim}
%%% 目次
\tableofcontents
\end{verbatim}
}
\end{breakbox}
特に気にせずとも上記命令のままで,目次が自動生成される.

\subsubsection{各章の読み込み}
ここから各章の記載である.
本パッケージでは,サンプルとして1章〜3章を読み込むようにしている.
具体的には \verb+\include+ 命令で chap1.tex chap2.tex chap3.tex が読み込まれている.
これらのファイル名は,適宜変更して構わない.
また,4章以降の部分はコメントアウトしているが,各自で適宜変更して欲しい.
\begin{breakbox}
{\small
%footnotesize
\begin{verbatim}
\chapter{はじめに}
本論文では,PCワークに限り,進捗管理と行動監視を行い,目標達成を支援するシステムについて記述する.

まず,本研究をおこなう背景となった事柄について述べる.
次に,研究目的の詳細を記述した後,類似研究との相違や関連研究とのつながりについて解説する.
また,次章以降の本論文の構成についてその概略を述べる.

\section{背景}
現在,勉強やオフィスワークでもPCを始めとするデジタル機器を利用することが一般的となっている.
それにより,今まで一般的であったオフィスなど,ひとつの拠点に集まって仕事をするのではなく,ひとりひとりが拠点を離れて仕事をすることも可能となった.在宅勤務は,育児や介護などのために自宅を離れられない個人にとって,家庭生活と仕事を両立するための手段として期待されている.場所の制約をなくすことは,出勤時間そのものも削減することができ,効率的に時間を使うことができる利点がある.

%以降削除すること
\clearpage
\noindent
一二三四五六七八九零一二三四五六七八九零一二三四五六七八九零一二三四五\\
二\\
三\\
四\\
五\\
六\\
七\\
八\\
九\\
零\\
一\\
二\\
三\\
四\\
五\\
六\\
七\\
八\\
九\\
零  行数と列数の設定テスト 30行×35文字 = 1050文字/ページ\\
一\\
二\\
三\\
四\\
五\\
六\\
七\\
八\\
九\\
零

 % 1章
\chapter{進捗監視システム}

本章では,PCワークにおける進捗監視システムの現状と,進捗監視を行うにあたり必要なプロセス監視方式について述べる.

\section{現在の監視システムについて}

ここでは,現在利用されている監視システムの例と,問題点について述べる.

\subsection{監視システムの例}
PCを利用したユーザの行動を監視する場面の一例として,オンライン試験が挙げられる.
インターネットの普及により,紙媒体による試験からPCによるオンライン試験へと移行が進んでいる.
オンライン試験は,試験監督の監視下にあるPCを用いて,不正行為が検出可能な状況で行われる必要がある.
そのため,受験者のPCに公正さを保つためのアプリケーションを導入することにより,オンライン試験が実現されている.

オンライン試験で利用されるアプリケーションでは,受験者の動作検出や回答状況の確認などを行うことができる.
受験者をリアルタイムで監視しておくことにより,公正に試験を受けているという証明を行う.

\subsection{監視システムの問題点}
現状の監視システムにおいては,自動化が不十分であり,システム自体が監視対象とコンピュータのみで完結させることはできていない.
先ほどの例では,不正行為を検知を監督者が目視で行うことになっている.
PC自体のプロセスを監視し,不正な動作を行っていないかどうかをプログラムによって判定するシステムも存在するが,100%不正行為を検知することはできない.

\section{監視方式}
PCの監視方式は大まかに,エージェントレス型と,エージェントレス型に分類される.
エージェントレス型は,特定のプログラムを監視対象のPC内で動作させることなく不正プロセスなどを監視する方式である.
エージェントレス型は,特定のアプリケーションを対象のPCにインストールさせて監視する方式である.

\subsection{エージェントレス型}
エージェントレス型の監視方式には,以下のようなものがある.

\subsubsection{リモートログイン}
ユーザに監視対象のPCから管理サーバへアクセスさせることで,監視を行う方法である.
監視される領域が管理サーバのアプリケーション内に限定される.
また,事前に各ユーザのアカウントを発行しておく必要がある.

\subsubsection{送受信パケット監視}
ネットワーク上を流れるパケットを監視することによって,不正な情報交換を監視する方法である.
監視対象が送受信パケットに限定されるため,外部にアクセスするようなプロセスしか監視できず,ネットワークを利用しないPC内に閉じたアプリケーションの動作は確認できない.

\subsection{エージェント型}
エージェント型の監視方式には,以下のようなものがある.

\subsubsection{インストール型}
監視対象のPCに監視用のアプリケーションを事前にインストールしておき,監視を行う方法である.
専用のデバイスではなく,各個人のPCを対象とする場合,自信を監視するソフトウェアをインストールさせること自体が難しい場合もある.

\subsubsection{エージェント送り込み型}
Webブラウザなどを使用し,アプリケーションの追加機能としてダウンロードさせることで,監視させる方法である.
インストールの操作が不要であるため,インストール型に対して抵抗感は少なくなる.

%以降削除すること
\clearpage
\noindent
一二三四五六七八九零一二三四五六七八九零一二三四五六七八九零一二三四五\\
二\\
三\\
四\\
五\\
六\\
七\\
八\\
九\\
零\\
一\\
二\\
三\\
四\\
五\\
六\\
七\\
八\\
九\\
零  行数と列数の設定テスト 30行×35文字 = 1050文字/ページ\\
一\\
二\\
三\\
四\\
五\\
六\\
七\\
八\\
九\\
零

 % 2章
\chapter{システム構成}

\section{システム概要}

\section{予定管理機能}

\section{アプリケーションによる監視}

\subsection{起動状況の記録}

\subsection{サービス検出}

\subsection{ユーザへの通知}

\section{作業記録の生成}

%以降削除すること
\clearpage
\noindent
一二三四五六七八九零一二三四五六七八九零一二三四五六七八九零一二三四五\\
二\\
三\\
四\\
五\\
六\\
七\\
八\\
九\\
零\\
一\\
二\\
三\\
四\\
五\\
六\\
七\\
八\\
九\\
零  行数と列数の設定テスト 30行×35文字 = 1050文字/ページ\\
一\\
二\\
三\\
四\\
五\\
六\\
七\\
八\\
九\\
零

 % 3章
%\chapter{実験}

\section{目的}

\section{対象}

\section{前提条件}

\subsection{達成目標}
ユーザの状況,達成目標,具体的に何の作業をやるのか

\subsection{作業の定義}
使用するアプリケーション,当実験に限定している部分を記述

\section{方法}

\section{結果}

\section{考察}

%以降削除すること
\clearpage
\noindent
一二三四五六七八九零一二三四五六七八九零一二三四五六七八九零一二三四五\\
二\\
三\\
四\\
五\\
六\\
七\\
八\\
九\\
零\\
一\\
二\\
三\\
四\\
五\\
六\\
七\\
八\\
九\\
零  行数と列数の設定テスト 30行×35文字 = 1050文字/ページ\\
一\\
二\\
三\\
四\\
五\\
六\\
七\\
八\\
九\\
零

 % 4章
%\chapter{おわりに}

\section{まとめ}

\section{今後の課題}

%以降削除すること
\clearpage
\noindent
一二三四五六七八九零一二三四五六七八九零一二三四五六七八九零一二三四五\\
二\\
三\\
四\\
五\\
六\\
七\\
八\\
九\\
零\\
一\\
二\\
三\\
四\\
五\\
六\\
七\\
八\\
九\\
零  行数と列数の設定テスト 30行×35文字 = 1050文字/ページ\\
一\\
二\\
三\\
四\\
五\\
六\\
七\\
八\\
九\\
零

 % 5章
%\include{chap6} % 6章
\end{verbatim}
}
\end{breakbox}

なお,これらのファイルは通常の\verb+\chapter+など \LaTeX の命令でマークアップしていけば良い.
chapter1.tex や chapter2.tex,chapter3.tex 内を見れば,おおよその方法は理解できるはずである.

\subsubsection{付録の設定と読み込み}
付録は以下の様になっている.
\begin{breakbox}
{\small
%footnotesize
\begin{verbatim}
%%% 付録 -- 必要なければ以下を2行コメントアウト
\appendix
\chapter{プログラムの動作方法}
本研究にて用いたプログラムについて解説する.

\section{ファイル構成}
プログラムのフォルダ内は,主に4つのファイルから構成される.

ああああいいいい

ううううええええ

これらを○○に設置し,以下の手順にそって起動する.

\section{起動方法}
まず,ウェブサーバを動かした状態にし,外部クライアント(Webブラウザから),以下のURLにアクセスする.


\section{表示の見方}
実験に利用するための,実行結果は test.log ファイルに出力されている.

このファイルは4つのカラムからなる CSV形式のファイルである.
第1列には,…





%\include{appendixB} %必要に応じて付録の数を増やす
\end{verbatim}
}
\end{breakbox}
サンプルとして 付録A(appendixA.tex)だけ読み込む様にしている.
このファイルも通常の\verb+\chapter+など通常の\LaTeX の命令でマークアップしていけば良い.
また,必要に応じて追加,コメントアウトして構わない.

\subsubsection{参考文献の設定と読み込み}
最後に参考文献の設定がなされている.
\begin{breakbox}
{\small
%footnotesize
\begin{verbatim}
\bibliographystyle{junsrt}
\bibliography{myrefs}
\end{verbatim}
}
\end{breakbox}
\verb+\bibliography{myrefs}+によって myrefs.bib ファイルが読み込まれている.
このファイルは \BibTeX のフォーマットにて記載されている.
詳細は3章にて記述する.
